\documentclass{article}
\usepackage[left=1.0in,right=1.0in,top=1.0in,bottom=1.0in]{geometry}
\usepackage{fancyhdr}
\usepackage{color}
\usepackage{amsmath}
\definecolor{light-gray}{gray}{0.5}


\begin{document}

\begin{center}
    {\LARGE \bf Point-by-Point Response to Reviewer's Comments} \\
     Author's Responses in {\color{red} Red}
\end{center}

%%%% First Reviewer %%%%
\section{Re: Information to the authors}
%
\subsection*{Larger issues}
%
\begin{enumerate}
    \item
         Some parts of the manuscript suggest that the authors introduced a new method,
         i.e., the complex-step (CS) method. For example, in the introduction, they say
         “A goal of this study was to develop and evaluate a new, accurate, and practical
         method ...” Another example appears in page 3, where the authors say “The aim
         of this paper is not only to introduce the new complex-step method in the context
         of evaluating tangent-stiffness matrices ..”. However, the CS method has already
         appeared in many publications since 1998 [1]. Please revise the way you refer to
         the CS method along the manuscript

{\color{red}
  \begin{itemize}
      \item
        The way the CS method is referred to has been revised throughout,
        specifically: 
    \item In the abstract, "... the new method ..." has been
        replaced by "... the method ...", "... newly implemented ..." has been
        replaced by "... implemented ..."; 
    \item in the introduction, paragraph two,
        "... a new, accurate ..." has been replaced by "... a relatively
        underexplored, accurate ...", in paragraph two and four "... the new
        method ..." has been replaced by "... the method ...", in paragraph
        five, "... introduce the new complex-step method in the context of ..."
        has been replaced by "... study the complex-step method of ...", "...
        the new complex-step ..." has been replaced by "... the complex-step
        ...".
    \item 
        It is the authors' opinion that the above changes correct the suggestion
        that the authors introduce a new method. 

  \end{itemize}}

    \item
        The work presented focuses on a parallel computational peridynamic code, and
        uses peridynamic models for mechanics. However, no detailed explanation of the
        equations involved is presented. Although the tangent-stiffness calculations are
        not specialized to Peridigm, the authors should include a short section introducing
        the peridynamic theory and the specific constitutive model used, i.e., the elastic
        peridynamic solid. You may add this to the already included very brief description
        of peridynamics in page 9 (lines 15-26). This will also help to explain the concept
        of the peridynamic horizon in page 10.

{\color{red}
  \begin{itemize}
    \item 
        We have revised the brief desciption of peridynamics in page 9 to be more complete:
    \item
        Material from silling2007 is interpreted including: the definition of vector states,
        the derivation of conservation laws in terms of state-based peridynamics, the definition
        of the isotropic elastic solid AKA the Linear elastic Peridynamic Solid.
    \item
        It is the authors' opinion that the additional material mentioned above explains the equations involved
        in the peridynamic models for mechanics, used in relation to the peridynamics code
        studied in the work. 


  \end{itemize}}




\end{document}
