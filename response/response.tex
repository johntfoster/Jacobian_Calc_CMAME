\documentclass{article}
\usepackage[left=1.0in,right=1.0in,top=1.0in,bottom=1.0in]{geometry}
\usepackage{fancyhdr}
\usepackage{color}
\usepackage{amsmath}
\definecolor{light-gray}{gray}{0.5}


\begin{document}

\begin{center}
    {\LARGE \bf Point-by-Point Response to Reviewer's Comments} \\
     Author's Responses in {\color{red} Red}
\end{center}

%%%% First Reviewer %%%%
\section{Information to the authors}
%
\subsection*{Larger issues}
%
\begin{enumerate}
    \item
         Some parts of the manuscript suggest that the authors introduced a new method,
         i.e., the complex-step (CS) method. For example, in the introduction, they say
         “A goal of this study was to develop and evaluate a new, accurate, and practical
         method ...” Another example appears in page 3, where the authors say “The aim
         of this paper is not only to introduce the new complex-step method in the context
         of evaluating tangent-stiffness matrices ..”. However, the CS method has already
         appeared in many publications since 1998 [1]. Please revise the way you refer to
         the CS method along the manuscript

{\color{red}
  \begin{itemize}
    \item We have added a paragraph at the end of Section 1 that highlights the difference between peridynamics and previous Òparticle methods.Ó The main difference is that in peridynamics, we numerically approximate a different set of continuum equations.
  \end{itemize}}





\end{document}
