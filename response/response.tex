\documentclass{article}
\usepackage[left=1.0in,right=1.0in,top=1.0in,bottom=1.0in]{geometry}
\usepackage{fancyhdr}
\usepackage{color}
\usepackage{amsmath}
\definecolor{light-gray}{gray}{0.5}


\begin{document}

\begin{center}
    {\LARGE \bf Point-by-Point Response to Reviewer's Comments} \\
     Author's Responses in {\color{red} Red}
\end{center}

%%%% First Reviewer %%%%
\section{Reviewer}
%
\subsection*{General Statement}
In general, I am not convinced of the benefits of particle methods when compared to traditional finite element analysis. Although they eliminate the sometimes troublesome problem of mesh generation, they create new problems of their own. In particular, they are hard to debug (because the patch test often doesn�t work), it is typically difficult to pin down an order of accuracy, and it is typically difficult to apply boundary conditions. Nonetheless, I recognize that they are part of the toolbox and therefore are within the scope of IJNME.
%
{\color{red}
  \begin{itemize}
    \item We have added a paragraph at the end of Section 1 that highlights the difference between peridynamics and previous �particle methods.� The main difference is that in peridynamics, we numerically approximate a different set of continuum equations.
  \end{itemize}}

This particular paper, however, has insufficient novelty to merit publication in its present form. In particular, 

\begin{enumerate}
  \item The peridynamic framework, including the method to compute the deformation gradient, is already established in the previous literature.
  {\color{red}
    \begin{itemize}
      \item We have added statements in the abstract and introduction that highlight what is new about the work being reported. Specifically, this represents the first successful implementation of a rate-dependent material model within peridynamics. We also present the first full-scale 3D solution of a realistic, reasonably complex application. This work therefore demonstrates to the community, for the first time, the ability of the peridynamic method to solve practical problems involving complex materials resulting in good agreement with experimental data.
    \end{itemize}}
  %
  \item The viscoplastic formulation and constitive model follows fairly classical nonlinear solid mechanics, and therefore does not appear to be novel. 
  {\color{red}
    \begin{itemize}
      \item Viscoplasticity has never been used within peridynamics.  See previous responses.
    \end{itemize}}
  %
  \item The computational studies are quite limited, by the authors� own admission (�while failure can be modeled with peridynamics, it is beyond the scope of this paper; therefore, these experiments were not attempted in numerical simulation�, p. 19). 
  {\color{red}
  \begin{itemize}
    \item The focus of this paper was to show that peridynamics could be used with conventional (stress-strain) constitutive models.   Modeling failure is the main advantage of the peridynamic approach, which has been demonstrated in the cited papers and will be further developed in a future paper with the combination of a viscoplastic constitutive model and a novel failure mechanism applied by breaking ``bonds''.
  \end{itemize}}
  %
\end{enumerate}
  
\subsection*{Specific Comments}

\begin{enumerate}
%
   \item p. 3, It has been proved that Equation 1 reduces to the classical continuum partial differential equation\dots Which one? Momentum?
   {\color{red}
     \begin{itemize}
        \item Yes, linear momentum.  This has been made more clear in the rewritten introduction.
     \end{itemize}}
%
  \item p. 5, should be $F = I + \nabla_x u$ (standard notation)
   {\color{red}
     \begin{itemize}
        \item The notation I used is identical to the notation used by Lawrence E. Malvern in his classic book \underline{Introduction to the Mechanics of a Continuous Medium} (p.\ 162 in the version I have).  This is further clarified by showing the incidental notation in the revision.
     \end{itemize}}
%
  \item Eq. (30) is an additive decomposition of the deviator of the referential tensor $\mathbf{d}$ based on the additive decomposition of the rate of deformation $\mathbf{D}$. A reference and/or some background should be included.
  {\color{red}
     \begin{itemize}
        \item Eq.\ (30) is an additive decomposition of the deviatoric part of $\mathbf{d}$ which is referred to as the {\em unrotated rate of deformation tensor}.  $\mathbf{d}$ is found from Eq.\ (17) by transforming $\mathbf{D}$, the {\em rate of deformation tensor}, with the rotation tensor $\mathbf{R}$.  This entire process is taken from the paper by Flanagan and Taylor ``An accurate numerical algorithm for stress integration with finite rotations'', which I believe to be a standard algorithm for developing kinematics in large displacement problems.
     \end{itemize}}
%
     \item Notation: Usually lower case letters are used for spatial quantities and upper case letters are used for referential quantities, e.g., $\mathbf{d}$ is standard for the rate of deformation tensor.
  {\color{red}
     \begin{itemize}
        \item All of the notation used in this paper follows very closely the notation used by Malvern and/or Flanagan and Taylor.
     \end{itemize}}
%
     \item I don't understand why eq.\ (35) is presented. As far as I know, the use of previous values of strain in order to extrapolate the strain rate is never used in the literature, and all previous authors obtain the strain by solving an equation like (36). So unless there is some good reason to believe that (35) might be plausible (e.g., an instance where it is successfully used in the previous literature), I would drop it from the paper.
  {\color{red}
     \begin{itemize}
        \item For the reviewer's reference a paper from Brown University by Peirce, Shih, and Needleman titled ``A tangent modulus method for rate dependent solids'' uses a method similar to what is presented in Eq.\ (35), a fully explicit method.
        \end{itemize}}
%
        \item Mistaken forward reference to eq. 59 on p. 14.
  {\color{red}
     \begin{itemize}
        \item Corrected in revision
     \end{itemize}}
%
      \item The paper mixes up modeling with discretization. For example, the model apparently doesn't discuss how the pressure component of stress is handled, and yet this pops up in the discussion of the numerics (eq.\ (51)).
  {\color{red}
     \begin{itemize}
        \item I do not fully understand this comment, von Mises plasticity only deals with the deviatoric part of stress to define a yield surface, but the pressure component must be added back in after the deviatoric stress has been updated. Nevertheless, a mistake was found in Eq.\ (51) and corrected in the revision.
      \end{itemize}}
%
        \item The total strain is apparently defined on a particle-by-particle basis using (61) and (12). I suppose the plastic strain is a nodal history variable?
  {\color{red}
     \begin{itemize}
        \item All stress or strain quantities including the internal state variable, {\em equivalent plastic strain}, are calculated for each particle at each timestep. 
     \end{itemize}}
%
        \item The number of nodes required seems to be huge in both simulations in Section 10. Is there any compensating benefit (versus finite elements)?
  {\color{red}
     \begin{itemize}
        \item The number on nodes was large, but these simulations ran in only a few hours on 4 processors.  The largest compensating benefit is the ability to model failure (outside the scope of this paper, but discussed in others) with the same set of fundamental equations without the need for separate governing equations for crack initiation, propagation, etc. (e.g., cohesive zones, element death).
     \end{itemize}}
 %
        \item The literature review includes previous papers (mainly by the same group of authors) on peridynamics but does not review the larger universe of particle methods for solid mechanics, and I think some references in this regard would be helpful. In particular, I think some of the previous particle methods are quite similar to peridynamics. Unfortunately, I am not an expert in this, but I know of at least three previous papers proposing particle methods for solid mechanics.
       \begin{enumerate}
         \item Rabczuk, Belytschko and Xiao, Stable particle methods based on Lagrangian kernels, CMAME 193 (2004) 1035-1063.
         \item Delaplace and Ibrahimbegovic, Performance of time-stepping schemes for discrete models in fracture dynamics, IJNME 65 (2005) 1527-1544.
         \item Gao and Klein, Numerical simulation of crack growht in an isotropic solid with randomized internal cohesive bonds, JMPS 46 (1998) 187-218.
       \end{enumerate}
%
     {\color{red}
     \begin{itemize}
        \item The main reason other particle methods were not reviewed was because all of these other methods are based on theory which solves partial differential equations and are therefore fundamentally different from the nonlocal integral equations solved in peridynamics.  However, additional discussion was added in the revision which points out these differences and references the reviewers suggested papers where appropriate. 
         \end{itemize}}
        %
        \item Inconsistent verb tense in second paragraph of summary (is, was.) 

p. 2, Where $\rho \rightarrow$ where $\rho$ 

Van der Walls $\rightarrow$ Van der Waals 

any pair of particles interact $\rightarrow$ interacts 

singe $\rightarrow$ single 

spacial $\rightarrow$ spatial 

principle $\rightarrow$ principal 

et. al $\rightarrow$ et al. (italicize)   
     
%
     {\color{red}
     \begin{itemize}
       \item Corrected in revision
     \end{itemize}}
%
\end{enumerate}

\section{Reviewer}
\subsection*{General Statement}
I recommend that the authors prepare a revised version of their paper to be considered for publication in the International Journal of Numerical Methods in Engineering.  Specifically, the authors should consider adding discussion related to the following eleven comments (1)-(11).
%
\subsection*{Specific Comments}

\begin{enumerate}

  \item This is the first time that I have read about the peridynamic theory.  In many ways it seems to be similar to the notions used in SPH (smooth particle hydrodynamics).  It is know that the SPH theories have difficulties handling material strength in a consistent stable manner.  I would expect the same problems with the peridynamic theory. It would be helpful to compare these theories and explain important differences and potential advantages. 
%
     {\color{red}
     \begin{itemize}
        \item Peridynamics is fundamentally different from SPH because SPH is still based on theory that solves partial differential equations.  In SPH, properties of each particle are smeared over a smoothing length, but the motion of the particles is still governed by PDE's.  In peridynamics the conservation of momentum equation is a nonlocal integral equation capable of handling even discontinuous displacements fields that would cause problems in SPH where the PDE's could not be evaluated.  SPH also suffers from a tensile instability that is not problematic for peridynamics.  Additional discussion has been added to the introduction to address these issues.
     \end{itemize}}
%
  \item Introducing nonlocality through the integral over the sphere of radius $\delta$ in the reference configuration violates the notion of a finite wave speed in the material.  Obviously the same problem occurs when finite elements are used to average deformation information over a finite region.  Some comments about the effect of this in this peridynamic theory would be helpful
     {\color{red}
     \begin{itemize}
        \item As the reviewer points out this is a similar issue in finite elements.  Wave speeds have been analyzed in peridynamics and have been shown to not have an infinite wave speed, but the wave speeds are a function of wave length.  When analyzed it turns out that the bulk modulus is directly related to $\delta$ and therefore, becomes an integral part of the defination of wave speed, i.e. $\delta$ is a material property.
     \end{itemize}}
%
  \item On page 2 it is stated that:
  \begin{quotation}
  \noindent ``In this implementation, $\omega$ is always either 0 or 1, which corresponds to whether the bond is `broken' or not.'' 
 \end{quotation}
 It seems to me that $\delta$ defines the radius of influence of material points and $\omega$ defines the strength of the influence.  In particular, it is surprising to me that a uniform distribution of $\omega$ with a sharp cut-off would not cause serious problems with wave propagation.  It would be helpful to comment on this in a revised version of the paper. 
     {\color{red}
     \begin{itemize}
        \item The reviewer is correct in his recognition of $\omega$ as a strength of influence.  So far only uniform distributions of $\omega$ have been evaluated and they do not show any numerical instablity.  Additional discussion was added to the revision.
     \end{itemize}}
%
  \item It would be helpful to provide a discussion of how displacement and traction boundary conditions can be implemented using this theory.
     {\color{red}
     \begin{itemize}
        \item Displacements, velocities, and body force densities can be applied to regions of finite volume.  Information on this was added to the revised article in the discussion of the numerical simulations.  See Silling ``Reformulation of elasticity theory for discontinuities and long-range forces'' cited in article for further reading.
     \end{itemize}}
%
  \item It would be helpful to discuss the specific approximations that are used when the sphere of influence (associated with $\delta$) extends outside of the boundaries of the body. 
     {\color{red}
     \begin{itemize}
        \item There are no issues when $\delta$ extends outside the boundaries of the body.  A proof has been included which shows that the deformation gradient can be correctly calculated for any assumed deformation state in the revised article.
     \end{itemize}}
 %
  \item It is my understanding that this method tracts the motion of material points. Since the sphere of influence (associated with $\delta$) is defined relative to the reference configuration, the same material points remain in communication even if the material is highly stretched or highly compressed. It would seem to me that for the highly stretched case the method would yield unphysical dispersion of waves since the region of influence can be quite large.
     {\color{red}
     \begin{itemize}
        \item The reviewer's assumptions are correct, but the same thing could happen in a finite element simulation as well when elements become severely stretched or distorted.  It would be possible to redefine the $\delta$ at every time step where material points far away would be removed from this horizon and others added; however, this adaptivity feature does not currenlty exist in the code used here, Emu.
     \end{itemize}}
 %
  \item For an elastic solid, which has a unique unstressed reference shape, it makes sense to refer the constitutive response to a reference configuration.  In contrast, the material response of an elastic-viscoplastic material, which has no unique unstressed configuration or shape, should be formulated in terms of constitutive equations which are explicitly independent of the arbitrary choice of the reference configuration. Moreover, since the main motivation for developing meshless methods is to handle large deformations it is unfortunate that the elastic-viscoplastic constitutive equations have been formulated with respect to the reference configuration using the rotation tensor $\mathbf{R}$ in (15).  Also, the integral operators are specified by considering regions defined in the reference configuration.  It is possible to develop this method in an Eulerian framework more similar to SPH methods? 
     {\color{red}
     \begin{itemize}
       \item The method of using $\mathbf{R}$ to provide an unrotated configuration in which to apply the constitutive model such that the stress rates will be objective is used in many finite element codes, including the production code at Sandia National Laboratories, PRESTO.  This method is described in detail in the Flanagan and Taylor paper ``An accurate numerical algorithm for stress integration with finite rotations.''  Futhermore, Eulerian frame constitutive equations can be used in peridynamics as long as we define $\mathbf{F}$ and $\mathbf{R}$.
     \end{itemize}}
     %
  \item The constitutive equations used for determining viscoplastic response are not clear. It is my understanding that (28) is a yield function which limits the magnitude of the deviatoric stress during loading
  \begin{equation}
    S = | \mathbf{S} | = \sqrt{\frac{2}{3}} G \quad \mbox{during loading.}
    \tag{28}
  \end{equation}
  Then, using the additive decomposition (29), the elastic constitutive equation (31), the flow rule (32) and (33), and the expression (38) it can be shown that 
  \begin{equation}
    \dot{S} = 2 \mu \left ( \mathbf{Q} : \dot{\mathbf{e}} - \sqrt{\frac{3}{2}} \dot{\varepsilon}_p \right ) = \sqrt{\frac{2}{3}} \dot{G}
    \tag{A1}
  \end{equation}
  However, viscoplasticity is introduced in by taking the value of $G$ to be a function of the equivalent plastic strain $\varepsilon_p$ and plastic strain rate $\dot{\varepsilon}_p$
  \begin{equation}
    G = f \left ( \varepsilon_p ,  \dot{\varepsilon}_p \right ) .
    \tag{34}
  \end{equation}
  Thus, substituting (34) into (A1) yields
  \begin{equation}
    2 \mu \left ( \mathbf{Q} : \dot{\mathbf{e}} - \sqrt{\frac{3}{2}} \dot{\varepsilon_p} \right ) = \left ( \frac{\partial f}{\partial \varepsilon_p} \right ) \dot{\varepsilon}_p + \left ( \frac{\partial f}{\partial \dot{\varepsilon}_p} \right ) \ddot{\varepsilon}_p
    \tag{A2}
  \end{equation}
  which is a second order differential equation for the equivalent plastic strain.  It is not clear that the proposed procedure described in sections 6 and 7 actually solve this second order equation for $\varepsilon_p$.
     {\color{red}
     \begin{itemize}
       \item What is derived in the paper is the finite differencing scheme for solving for an increment of equivalent plastic strain.  There is no need to consider the continous derivatives when we do this.  Nevertheless I will start with equation (A1) and show that it equals the final result from the paper.  Taking the right-hand equality from (A1) we can substitute (39)\footnote{\color{red} Equation numbers refer to equation numbers in original paper, not the revision} for $\mathbf{Q:\dot{e}}$. Noting that because $\mathbf{Q}$ and $\mathbf{\dot{e}}$ are symmetric tensors $\mathbf{Q:\dot{e}} = \mathbf{\dot{e}:Q}$.
        \begin{equation}
	  2 \mu \left ( \dot{e} - \sqrt{\frac{3}{2}} \dot{\varepsilon}_p \right ) = \sqrt{\frac{2}{3}} \dot{G}
	    \tag{B1}
	  \end{equation}
	  Now substitute (38) and (41)
	  \begin{equation}
	    2 \mu \dot{e}^e = \sqrt{\frac{2}{3}} \dot{G}
	    \tag{B2}
	  \end{equation}
	  Convert from continous derivatives to rates.
	  \begin{equation}
	    2 \mu \frac{\Delta e_{t-\Delta t/2}^e}{\Delta t} = \sqrt{\frac{2}{3}} \frac{\Delta G_{t-\Delta t/2}}{\Delta t}
	    \tag{B3}
	  \end{equation}
	  Multiply by $\Delta t$ and expand the RHS of (B3)
	  \begin{equation}
	    2 \mu {\Delta e_{t-\Delta t/2}^e} = \sqrt{\frac{2}{3}} \left ( G_{t}- G_{t-\Delta t} \right )
	    \tag{B4}
	  \end{equation}
	  and of course $G_t = \sqrt{3/2} S_t$ and similary for $G_{t-\Delta t}$, using these in (B4)
	  \begin{equation}
	    2 \mu {\Delta e_{t-\Delta t/2}^e} = S_t - S_{t-\Delta t}
	    \tag{B5}
	  \end{equation}
	  finally substituting the rate form of (41) and rearranging we have:
	  \begin{equation}
	    \Delta e_{t-\Delta t/2} - \frac{S_t - S_{t-\Delta t}}{2 \mu} - \Delta \lambda_{t-\Delta t/2} = 0
	    \tag{B6}
	  \end{equation}
	  (B6) is exactly (44).  The rest of derivation follows the paper from (44).
     \end{itemize}}
 %
  \item Line 7 after (8).  $\mathbf{K}$  is a second order moment not a first order moment since $\boldsymbol{\xi} \otimes \boldsymbol{\xi}$ is a quadratic term.
     {\color{red}
     \begin{itemize}
        \item The description of $\mathbf{K}$ as a moment has been removed in the revision.
     \end{itemize}}
     %
  \item The value of $\delta$ used for the simulations should be specified.  In this regard, it would be good to also state how many material points used in the simulation fit into this sphere of radius $\delta$. 
     {\color{red}
     \begin{itemize}
        \item Suggestion was included in revision along with additional discussion about how changing $\delta$ makes insignificant changes in the solution of the Taylor impact problem.
     \end{itemize}}
        %
  \item Figure 2 refers to true strain but the measure $\mathbf{E}$ in (61) is the Lagranian strain.  Please provide a definition of the strain measure being used for true strain in Fig. 2. 
     {\color{red}
     \begin{itemize}
        \item Suggestion was included in revision.
     \end{itemize}}
\end{enumerate}


\end{document}
