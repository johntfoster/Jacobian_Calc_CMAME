\documentclass{article}
\usepackage[left=1.0in,right=1.0in,top=1.0in,bottom=1.0in]{geometry}
\usepackage{fancyhdr}
\usepackage{color}
\usepackage{amsmath}
\definecolor{light-gray}{gray}{0.5}


\begin{document}

\begin{center}
    {\LARGE \bf Point-by-Point Response to Reviewer's Comments} \\
     Author's Responses in {\color{red} Red}
\end{center}

%%%% First Reviewer %%%%
\section{Re: Information to the authors}
%
\subsection*{Larger issues}
%
\begin{enumerate}
    \item
         Some parts of the manuscript suggest that the authors introduced a new method,
         i.e., the complex-step (CS) method. For example, in the introduction, they say
         “A goal of this study was to develop and evaluate a new, accurate, and practical
         method ...” Another example appears in page 3, where the authors say “The aim
         of this paper is not only to introduce the new complex-step method in the context
         of evaluating tangent-stiffness matrices ..”. However, the CS method has already
         appeared in many publications since 1998 [1]. Please revise the way you refer to
         the CS method along the manuscript

{\color{red}
  \begin{itemize}
      \item
        The way the CS method is referred to has been revised throughout,
        specifically: 
    \item In the abstract, "... the new method ..." has been
        replaced by "... the method ...", "... newly implemented ..." has been
        replaced by "... implemented ..."; 
    \item in the introduction, paragraph two,
        "... a new, accurate ..." has been replaced by "... a relatively
        underexplored, accurate ...", in paragraph two and four "... the new
        method ..." has been replaced by "... the method ...", in paragraph
        five, "... introduce the new complex-step method in the context of ..."
        has been replaced by "... study the complex-step method of ...", "...
        the new complex-step ..." has been replaced by "... the complex-step
        ...".
    \item 
        It is the authors' opinion that the above changes correct the suggestion
        that the authors introduce a new method. 

  \end{itemize}}

    \item
        The work presented focuses on a parallel computational peridynamic code, and
        uses peridynamic models for mechanics. However, no detailed explanation of the
        equations involved is presented. Although the tangent-stiffness calculations are
        not specialized to Peridigm, the authors should include a short section introducing
        the peridynamic theory and the specific constitutive model used, i.e., the elastic
        peridynamic solid. You may add this to the already included very brief description
        of peridynamics in page 9 (lines 15-26). This will also help to explain the concept
        of the peridynamic horizon in page 10.

{\color{red}
  \begin{itemize}
    \item 
        We have revised the brief description of peridynamics in page 9 to be more complete:
    \item
        Material from silling2007 is interpreted including: the definition of vector states,
        the derivation of conservation laws in terms of state-based peridynamics, the definition
        of the isotropic elastic solid AKA the Linear elastic Peridynamic Solid.
    \item
        It is the authors' opinion that the additional material mentioned above explains the equations involved
        in the peridynamic models for mechanics, used in relation to the peridynamics code
        studied in the work. 

  \end{itemize}}
    \item 
        The manuscript seems to be written in a language more familiar to computer scientists. 
        I would encourage the authors to revise certain parts of the manuscript to be
        presented in a language more familiar to the computational mechanics community.

{\color{red}
  \begin{itemize}
    \item 
        We have revised certain parts of the manuscript for clarity, adding additional information for the reader's
        convenience:
    \item
        In subsection 'Implementing the complex-step method in Peridigm', explanation of what is meant by datatype,
        container class, template, scope and instantiation are included since these concepts may be unfamiliar to
        a computational mechanics audience at least as denoted by these terms. 
    \item 
        In subsection 'The Comparative Study', the way message passing computations can slow runtime is explained
        by an analogy to an experiment given as an exercise for the reader. 
    \item 
        In section 'Conclusions', the term byte-copiable is explained. 
    \item 
        It is the authors' opinion that the above revisions to the language of the manuscript allow the intended audience
        to follow the discussion where concepts from computer science are relevant.
  \end{itemize}}

  \item
      The comparisons of the different algorithms are only performed with respect to the
      construction of tangent-stiffness matrices. In par ticular, the authors always use the
      solution of AD at each time step to compute the entries of the tangent-stiffness
      matrices, even for the FD, CD, and CS methods. However, the authors need to
      demonstrate the performance of the CS method, in comparison to the other ones,
      with respect to the solution of real, ideally non-linear problems. 

{\color{red}
  \begin{itemize}
    \item 
        To address the reviewers comment, material from work done after the
        initial submission of the manuscript has been included. New sections
        include 'Working examples of CS, AD and FD for Newton's method' with
        new subsections. To existing sections were appended new subsections
        including 'Convergence and displacement accuracy studies', 'For further
        studies of Jacobian accuracy in Newton's method', 'Grain of salt for
        speed results' and 'Remarks'. Specifically, the added material addresses
        the reviewers comments:
    \end{itemize}}

\item 
      In other words, how the solutions compare in terms of accuracy and speed between the methods when
      solving given problems? Please include two examples of interest. The authors
      suggest that two examples involving non-linear systems are already included in the
      paper repository.

{\color{red}
  \begin{itemize}
    \item 
        The added section 'Working examples of CS, AD and FD for Newton's method' includes
        the two examples alluded to in the manuscript and requested by the reviewer. 
        % 
        To summarize, in this section is an example of interest: a displacement accuracy study and 
        a convergence rate study on a non-linear beam problem, where each of the methods
        discussed in the manuscript are used individually to solve the problem. 
 \end{itemize}}

\item
      The authors say in the Conclusion: “While the results showed that CS
      produced more accurate tangent-stiffness matrices than CD and FD under
      the parameters of the tests, it was not determined whether or not this is
      a clear advantage of CS over FD in terms of accuracy of final predicted
      displacements and speed of convergence.” This needs to be studied and
      results in this regard added to this manuscript.

{\color{red}
  \begin{itemize}
    \item 
      The added subsection 'Convergence and displacement accuracy studies'
      includes convergence rate and displacement accuracy studies similar to
      those shown in 'Working examples ...', but with the models and tools of
      the main study. This subsection provides an example of interest. 
  \item 
      To summarize, these studies are done in \emph{Peridigm} using the most refined mesh
      from the 'serial' test runs from the original manuscript. Measured was the performance of the methods in terms of speed
      (convergence rate) and accuracy (displacement accuracy).  Iteration speed
      was reported in the original manuscript.
  \item
      It was determined that the accuracy of CS over FD resulted in no appreciable
      advantage in terms of displacement accuracy or convergence rate for the problems
      solved in this study. 
    \item
      Additional comment on the significance of the accuracy of the Jacobian
      to Newton's method is included in the subsections 'For further studies of
      Jacobian accuracy in Newton's method', 'Grain of salt for speed results'
      and 'Remarks'. 
  \item 
      To summarize, Newton's method is insensitive to the accuracy differences
      shown by the methods, for the problems shown, when the methods are not
      deliberately misused with inappropriate step sizes. For completeness,
      formulas describing a relationship between Jacobian accuracy and
      convergence rate order are included. 
  \item
      It is the authors' opinion that the above revisions offer the discussion of relative problem solving
      performance of the methods that the reviewer has requested, including consideration
      of speed and accuracy in solving real, non-linear problems.
  \end{itemize}}

  \item
     In Section 1.1.1, please include a more comprehensive literature review for
     the CS method, including fur ther references (some of them are listed below).
     In particular, specify what aspects of the CS method have been investigated in
     each reference. Then, you need to clearly mention your contribution within
     the CS method, in contrast to the other references (some comments in this
     regard appear in the manuscript in page 8).

{\color{red}
  \begin{itemize}
  \end{itemize}}



\end{document}
