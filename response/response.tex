\documentclass{article}
\usepackage[left=1.0in,right=1.0in,top=1.0in,bottom=1.0in]{geometry}
\usepackage{fancyhdr}
\usepackage{color}
\usepackage{amsmath}
\definecolor{light-gray}{gray}{0.5}


\begin{document}

\begin{center}
    {\LARGE \bf Point-by-Point Response to Reviewer's Comments} \\
     Author's Responses in {\color{red} Red}
\end{center}

%%%% First Reviewer %%%%
\section{Re: Information to the authors}
%
\subsection*{Larger issues}
%
\begin{enumerate}
    \item
         Some parts of the manuscript suggest that the authors introduced a new method,
         i.e., the complex-step (CS) method. For example, in the introduction, they say
         “A goal of this study was to develop and evaluate a new, accurate, and practical
         method ...” Another example appears in page 3, where the authors say “The aim
         of this paper is not only to introduce the new complex-step method in the context
         of evaluating tangent-stiffness matrices ..”. However, the CS method has already
         appeared in many publications since 1998 [1]. Please revise the way you refer to
         the CS method along the manuscript

{\color{red}
  \begin{itemize}
      \item
        The way the CS method is referred to has been revised throughout,
        specifically: 
    \item In the abstract, "... the new method ..." has been
        replaced by "... the method ...", "... newly implemented ..." has been
        replaced by "... implemented ..."; 
    \item in the introduction, paragraph two,
        "... a new, accurate ..." has been replaced by "... a relatively
        underexplored, accurate ...", in paragraph two and four "... the new
        method ..." has been replaced by "... the method ...", in paragraph
        five, "... introduce the new complex-step method in the context of ..."
        has been replaced by "... study the complex-step method of ...", "...
        the new complex-step ..." has been replaced by "... the complex-step
        ...".
    \item 
        It is the authors' opinion that the above changes correct the suggestion
        that the authors introduce a new method. 

  \end{itemize}}

    \item
        The work presented focuses on a parallel computational peridynamic code, and
        uses peridynamic models for mechanics. However, no detailed explanation of the
        equations involved is presented. Although the tangent-stiffness calculations are
        not specialized to Peridigm, the authors should include a short section introducing
        the peridynamic theory and the specific constitutive model used, i.e., the elastic
        peridynamic solid. You may add this to the already included very brief description
        of peridynamics in page 9 (lines 15-26). This will also help to explain the concept
        of the peridynamic horizon in page 10.

{\color{red}
  \begin{itemize}
    \item 
        We have revised the brief description of peridynamics in page 9 to be more complete:
    \item
        Material from silling2007 is interpreted including: the definition of vector states,
        the derivation of conservation laws in terms of state-based peridynamics, the definition
        of the isotropic elastic solid AKA the Linear elastic Peridynamic Solid.
    \item
        It is the authors' opinion that the additional material mentioned above explains the equations involved
        in the peridynamic models for mechanics, used in relation to the peridynamics code
        studied in the work. 

  \end{itemize}}
    \item 
        The manuscript seems to be written in a language more familiar to computer scientists. 
        I would encourage the authors to revise certain parts of the manuscript to be
        presented in a language more familiar to the computational mechanics community.

{\color{red}
  \begin{itemize}
    \item 
        We have revised certain parts of the manuscript for clarity, adding additional information for the reader's
        convenience:
    \item
        In subsection 'Implementing the complex-step method in Peridigm', explanation of what is meant by datatype,
        container class, template, scope and instantiation are included since these concepts may be unfamiliar to
        a computational mechanics audience at least as denoted by these terms. 
    \item 
        In subsection 'The Comparative Study', the way message passing computations can slow runtime is explained
        by an analogy to an experiment given as an exercise for the reader. 
    \item 
        In section 'Conclusions', the term byte-copiable is explained. 
    \item 
        It is the authors' opinion that the above revisions to the language of the manuscript allow the intended audience
        to follow the discussion where concepts from computer science are relevant.
  \end{itemize}}

  \item
     In Section 1.1.1, please include a more comprehensive literature review for
     the CS method, including further references (some of them are listed below).
     In particular, specify what aspects of the CS method have been investigated in
     each reference. Then, you need to clearly mention your contribution within
     the CS method, in contrast to the other references (some comments in this
     regard appear in the manuscript in page 8).

{\color{red}
  \begin{itemize}
     \item 
         The authors have appended a more comprehensive literature review for the CS method to section 1.1.1.
         The aspects of CS investigated by the prior work, as well as any particularly interesting
         developments introduced by those works, have been listed. To argue for the novelty of the instant work,
         each prior work is compared to the instant work in terms of the use of the CS method and the similarity of the fields of research of the works.
     \item
         The literature review concludes with an argument for the novelty of the instant
         work where the prior works are taken as a whole and compared to the instant work. It is the authors' opinion
         that the instant work is novel for at least the reasons set out in the literature review material, where
         the contribution to the CS method is in the comparison to other methods in terms of computational
         time,  parallel implementation, and implementation within \emph{Peridigm} .
  \end{itemize}}


  \item
      The comparisons of the different algorithms are only performed with respect to the
      construction of tangent-stiffness matrices. In particular, the authors always use the
      solution of AD at each time step to compute the entries of the tangent-stiffness
      matrices, even for the FD, CD, and CS methods. However, the authors need to
      demonstrate the performance of the CS method, in comparison to the other ones,
      with respect to the solution of real, ideally non-linear problems. 

{\color{red}
  \begin{itemize}
    \item 
        To address the reviewers comment, material from work done after the
        initial submission of the manuscript has been included. New sections
        include 'Working examples of CS, AD and FD for Newton's method' with
        new subsections. To existing sections were appended new subsections
        including 'Convergence and displacement accuracy studies', 'For further
        studies of Jacobian accuracy in Newton's method', 'Grain of salt for
        speed results' and 'Remarks'. Specifically, the added material addresses
        the reviewers comments:
    \end{itemize}}

\item 
      In other words, how the solutions compare in terms of accuracy and speed between the methods when
      solving given problems? Please include two examples of interest. The authors
      suggest that two examples involving non-linear systems are already included in the
      paper repository.

{\color{red}
  \begin{itemize}
    \item 
        The added section 'Working examples of CS, AD and FD for Newton's method' includes
        the two examples alluded to in the manuscript and requested by the reviewer. 
        % 
        To summarize, in this section is an example of interest: a displacement accuracy study and 
        a convergence rate study on a non-linear beam problem, where each of the methods
        discussed in the manuscript are used individually to solve the problem. 
 \end{itemize}}

\item
      The authors say in the Conclusion: “While the results showed that CS
      produced more accurate tangent-stiffness matrices than CD and FD under
      the parameters of the tests, it was not determined whether or not this is
      a clear advantage of CS over FD in terms of accuracy of final predicted
      displacements and speed of convergence.” This needs to be studied and
      results in this regard added to this manuscript.

{\color{red}
  \begin{itemize}
    \item 
      The added subsection 'Convergence and displacement accuracy studies'
      includes convergence rate and displacement accuracy studies similar to
      those shown in 'Working examples ...', but with the models and tools of
      the main study. This subsection provides an example of interest. 
  \item 
      To summarize, these studies are done in \emph{Peridigm} using the most refined mesh
      from the 'serial' test runs from the original manuscript. Measured was the performance of the methods in terms of speed
      (convergence rate) and accuracy (displacement accuracy).  Iteration speed
      was reported in the original manuscript.
  \item
      It was determined that the accuracy of CS over FD resulted in no appreciable
      advantage in terms of displacement accuracy or convergence rate for the problems
      solved in this study. 
    \item
      Additional comment on the significance of the accuracy of the Jacobian
      to Newton's method is included in the subsections 'For further studies of
      Jacobian accuracy in Newton's method', 'Grain of salt for speed results'
      and 'Remarks'. 
  \item 
      To summarize, Newton's method is insensitive to the accuracy differences
      shown by the methods, for the problems shown, when the methods are not
      deliberately misused with inappropriate step sizes. For completeness,
      formulas describing a relationship between Jacobian accuracy and
      convergence rate order are included. 
  \item
      It is the authors' opinion that the above revisions offer the discussion of relative problem solving
      performance of the methods that the reviewer has requested, including consideration
      of speed and accuracy in solving real, non-linear problems.
  \end{itemize}}
  \end{enumerate}

  \subsection*{Smaller issues}
  \begin{enumerate}

  \item
	Pg. 2: In line 14, explain what does the term “algorithmically consistent” mean.

{\color{red}
  \begin{itemize}
     \item

      A brief definition for the convenience of the reader was added in the
      manuscript. But, algorithmically consistent is a fairly common term in
      plasticity/mechanics. This same meaning is used for 'consistent' refering
      to 'tangent matrices' in \cite{perez2012numerical, perez2000numerical}. It means
      consistent with integrating the plasticity model under the Khun-Tucker
      constraints. Integrating the plasticity model means, given a strain
      increment, determining a stress that does not exceed the yield stress,
      and by doing so computing a linear slope, called a consistent tangent
      modulus, that relates the strain increment to a correct stress increment.
      An alogorithmically consistent tangent stiffness matrix is then one that
      uses the consistent tangent modulii, making it a more accurate
      approximation of the stiffness of the system. 

  \end{itemize}}

  \item
	Pg. 2: In line 36, what do you mean by “discrimination”?

{\color{red}  
\begin{itemize}
     \item

      The word, 'discrimination' has been removed from the manuscript.
      'Discrimination' was meant to mean 'measuring differences and making
      judgements on relative usefulness'. Since this meaning is probably
      understood from the word 'comparison' alone, the word 'discrimination'
      has been removed from the manuscript for clarity.
      
  \end{itemize}}
	
` \item	
	Pg. 2: In line 37, explain what do you mean by “in-situ instrumentation”.

{\color{red}  
\begin{itemize}
     \item
         
      The words, 'in-situ' have been removed from the manuscript. 'In-situ' was
      meant to suggest that timing measurements were taken from within and by
      the program being analyzed itself, rather than by an external program,
      but it has been removed for clarity.

  \end{itemize}}
	
  \item 
	Pg. 3: In lines 19-29, specify the sections where each part of the manuscript is
included.

{\color{red}  
\begin{itemize}
     \item
      The correction is made in the manuscript.
  \end{itemize}}

  \item 
Pg. 3: You may consider including the exact web address of the paper repository.

{\color{red}  
\begin{itemize}
     \item
      The correction is made in the manuscript. An exact web address for the paper repository has been added.
  \end{itemize}}

  \item 
    Pg. 3: Please include the forward and centered finite difference expressions for
    easier comparison with (2); include the truncation error as well. You do not
    have to derive these expressions though.

{\color{red}  
\begin{itemize}
     \item
      The correction is made in the manuscript. Expressions for forward and centered difference have been included for faster comparison with (2).  
  \end{itemize}}

  \item 
Pg. 4: In Eq. (1), “$\partial f^2$” should be “$\frac{\partial^2 f}{\partial x^2}$”; similarly, for the third derivative.

{\color{red}  
\begin{itemize}
     \item
      The correction is made in the manuscript.
  \end{itemize}}

  \item 
Pg. 4: You may consider writing Eq. (2) with the truncation error

{\color{red}  
\begin{itemize}
     \item
      The correction is made in the manuscript.
  \end{itemize}}

\item
Pg. 6: In lines 9-11, please add references to the “Poisson problem” and to the
“mathematical optimization”.

{\color{red}  
\begin{itemize}
     \item
         [TODO]
         
  \end{itemize}}

\item
Pg. 7: Should Eqs. (4), (5), and (6) use “$\approx$” instead of “$=$”?

{\color{red}  
\begin{itemize}
     \item
         Manuscript corrected. These expressions are approximations.
  \end{itemize}}

\item
Pg. 7: In line 28, note that (cid:126)u is not the current configuration, but the displacement
field. Please revise the sentence.

{\color{red}  
\begin{itemize}
    \item
        Manuscript corrected. Sentence was revised to mean that F, evaluated in the current configuration, is a function of the displacement rather than to say that the displacement is the current configuration.
        \end{itemize}}

\item
Pg. 7: Please provide an expression for ${F^int}_i$ in the model used.

{\color{red}  
\begin{itemize}
    \item
        [TODO]
        Manuscript corrected. In the section 'Peridigm and peridynamics' a constitutive model for the LPS is shown. On the page that was originally page 7, the force density corresponding to this constitutive model is expressed in descrete form.
  \end{itemize}}

\item
Pg. 9: In line 26, please add the reference [12] below to “coupling to molecular
simulations [29].”

{\color{red}  
\begin{itemize}
    \item
        The manuscript has been corrected.
  \end{itemize}}

\item
Pg. 10: Please add an illustration of the block undergoing tension in Section 2.2,
for clarity. Also, clarify the displacement boundary conditions; are they applied in a
volumetric layer?

{\color{red}  
\begin{itemize}
    \item
        The manuscript has been corrected. Viewgraphs of the 2000 node and 1000000 node blocks undergoing tension has been included. The images depict colorized meshes in a way that is used in the revised manuscript to help explain what boundary conditions or 'volume constraints' are like in peridynamics. Displacement boundary conditions are typically applied to a volumetric layer, and in the study shown in the manuscript that is they way they were applied.
  \end{itemize}}

\item
Pg. 10: In line 40, you mentioned a bandwidth of 7. This seems to be clear in 1D.
Does this hold in higher dimensions? Please clarify.

{\color{red}  
\begin{itemize}
    \item
        The manuscript has been corrected. Bandwidth was referring to connections between nodes rather than components and degrees of freedom. It is more clear to say a bandwidth of 21 for the 3D case. The formula is number of nodes in a neighborhood times degrees of freedom.
  \end{itemize}}

    \item
        In lines 47-49, you say “The specific parameters ... can be found ... in the
        paper repository.” It seems a good idea to have the paper repository with files that a
        user can eventually use. However, the content of the paper should be independent
        of the repository. Please include a table with the parameters used. Similarly, in
        page 11 (lines 11-13).

{\color{red}  
\begin{itemize}
    \item
        Correction made in the manuscript. 
  \end{itemize}}

  \item
    Pg. 14: In line 55, is there any conclusion you can draw from the power law index
    $1.0 * 10^-6$?

{\color{red}  
\begin{itemize}
     \item
      Manuscript updated. More significant than the power law index is that the methods have the same power law relationship with computation time at all, meaning that there is some weak evidence that the methods have the same complexity class. 
  \end{itemize}}

  \item
    Pgs. 14-: Please clarify the value of the meshsize used on each of the experiments.

{\color{red}  
\begin{itemize}
     \item
         Manuscript is corrected. Mesh sizes from smallest to largest were 1000, 2000, 3000, 4000, 6000, 8000, 12000,
         16000, 32000, 1000000 peridynamic nodes.
      \end{itemize}}

  \item
    Pg. 16: Would it be possible to rescale Fig. 2, so that the dependence of the time
    on the number of cores will look indeed linear?

{\color{red}  
\begin{itemize}
     \item
     Manuscript is in error and has been corrected. An accurate statement is that CPU time and number of cores are inversely related.
  \end{itemize}}

  \item
    Pg. 16: In lines 54-56, you say “It should be pointed out that compared to the (cid:96)2 -
    norms of any of the TS matrices by themselves, the magnitude of the accuracy
    metric ... is relatively small.” Please include in the manuscript the explicit informa-
    tion on the relative error. It seems that you already have such information in the
    paper repository.

{\color{red}  
\begin{itemize}
     \item
    The manuscript has been corrected. 
    \end{itemize}}

  \item
    Pg. 17: Include the values of the probe distances h used for each method.

{\color{red}  
\begin{itemize}
     \item
    Correction made in the manuscript, values used are collected and the reference to the material explaining these values was made more clear.
    %
    The probe distances used in Peridigm are automatically determined with a
    heuristic algorithm and are not set by the user. The algorithm for this in
    \emph{Peridigm} is the used for another Sandia software project called
    \emph{Adagio} \cite{ref-Adaggio}. The probe distances are 1.E-6 times the node spacing for FD and CD which is indended to prevent the onset of roundoff error.
    1.E-100 was hard-coded for CS into the CS method implementation, while AD had no probe distance.
  \end{itemize}}

  \item
    Pg. 17: In lines 53-54, you say “Accuracy trends matched those for the single core
    tests.” Indeed, comparing Figs. 3 and 4, the results for FD seem to be quite the
    same in serial and in parallel. For CS the accuracy seems higher in serial. For CD,
    the accuracy is also higher in serial and, in serial, it has a fluctuating behavior. Can
    you comment on these observations?

{\color{red}  
\begin{itemize}
     \item
     	Manuscript has been updatded. When accuracy is looked at in terms of relative error, where relative error is the accuracy metric previously plotted in the figures divided by the Frobenius norm of the AD TS matrix alone times 100, there is actually little fluctuation, between tests or between serial and parallel tests. The order of magnitude for this measure remains steady for all the methods.
  \end{itemize}}

  \item
    Pgs. 17-18: What are the units of K ? If F int in Eq. (4) etc. is internal force, then K
    should have units of force/length. However, in Figs. 3 and 4 you use the unit of
    MPa, which is force/length2 . Please clarify.

{\color{red}  
\begin{itemize}
     \item
    The manuscript has been corrected. The units of K are actually force / length4, however there is no intuitive relationship between the norm of the TS matrix and some physical characteristic of the system being modeled.
  \end{itemize}}

    \item
        Pg. 19: Are the units of the vertical axis in Fig. 5 “seconds”, as in Figs. 1 and 2?
        Please indicate that. Also, is the label of the vertical axis in Fig. 6 correct?

{\color{red}  
\begin{itemize}
     \item
        The manuscript has been corrected. The labels on the vertical axis in Fig. 5 were incorrect. 
  \end{itemize}}

\item
Pg. 20: In lines 32-35, clarify what “byte-copyable” mean and why is that important.

{\color{red}  
\begin{itemize}
     \item
     Clarification made in manuscript, new reference added. 
     %
     'Byte-copiable' means that members of an array of data elements are all of
     the same size in memory so that if one knew the position of the first
     element in the array, the position of all other elements in the array
     could be inferred with the additional information of the length of the
     array. This is important, in particular, to the designers and users of the
     thousands of MIC accelerator cards installed in the \emph{Stampede}
     computer system. MIC accelerator cards permit only the transfer of
     byte-copiable data from the host computer system for use with the '#pragma
     offload' mode, which is to say most code written for the MIC. This is even
     more important when it is realized that the great majority of
     \emph{Stampede}'s theoretical processing power, the figure used to advertise to potential clients, is counted from these MIC
     accelerator cards. Meaning that if an AD user wanted to use any of that
     power, they would be out of luck since not even the full time personel at
     TACC could manage to compile the AD library Trilinos::Sacado for the MIC.
     A more 'casual' computational scientist would find the task impossible.
  \end{itemize}}

 \item
    Pg. 21: Please include a section title for the references.

{\color{red}  
\begin{itemize}
     \item
         Manuscript corrected.
  \end{itemize}}

 \item
    Pg. 24: In line 23, you may add the truncation error O(h2 ) to the expression. Oth-
    erwise, you may need to replace the “=” sign by the “≈” sign. However, adding
    the error term would make more clear the limiting expression when h → 0. Also,
    replace the comma by a period.

{\color{red}  
\begin{itemize}
     \item
         Manuscript corrected.
  \end{itemize}}

\item    
Pg. 24: Why do we need S and X in Appendix B?

{\color{red}  
\begin{itemize}
     \item

     All the members of set X need to be members of set S otherwise the
     approximations don't work. X are numbers from the 'real' world and S are
     numbers from the computer world. For example:

     \begin{verbatim}
     "http://en.wikipedia.org/wiki/IEEE_floating_point" 
     \end{verbatim}
     
     shows that computers
     and languages following IEEE standards don't represent very large numbers
     with very fine precision at the same time due to memory constraints, and
     there are upper and lower bounds on the magnitudes of nonzero numbers that
     can be represented. Obviously, approximation error is expected, but the
     fact that a user doesn't actually get 'all' the numbers is something that
     is not obvious.

  \end{itemize}}


\item
    Pgs. 25-26: Are the functions in the second column of the table in 2(c) (page 25)
    approximations? Why do you label the column “Approximates Function”? Simi-
    larly, in line 16 (page 26) you say “approximated by the partial derivative”; is it an
    approximation?

{\color{red}  
\begin{itemize}
     \item

      Strictly speaking, a computer can only provide finite precision
      approximations, of approxmations, of functions. Approximations of
      derivatives of functions are also approximations, therefor the label is
      correct in both cases. For example:

      \begin{verbatim}
      "https://sourceware.org/git/?p=glibc.git;a=blob;f=sysdeps/x86_64/fpu/e_expl.S"
      \end{verbatim}

      is the implementation of the $exp(x)$ function on 80-bit extended
      precision floating point numbers in the popular GCC compiler. This
      implementation is the most accurate that GCC provides for the $e$
      function on real numbers, but ultimately is still a discrete numerical
      algorithm.

  \end{itemize}}

  \item
Pg. 26: Please clarify whether storage is needed in automatic differentiation. For
example, do we need to store w, v , and u?

{\color{red}  
\begin{itemize}
     \item
      Clarification is added in the manuscript.  
  \end{itemize}}


  \end{enumerate}

\bibliographystyle{elsarticle-num}
\bibliography{all}
\end{document}
