\documentclass[10pt,letterpaper,draft]{article}
\usepackage[utf8]{inputenc}
\usepackage{amsmath}
\usepackage{amsfonts}
\usepackage{amssymb}
\author{Michael D. Brothers}
\title{Comparing and contrasting with prior work in the field}
\begin{document}
\section{Abstract of the reviewed work}

In order to maintain the quadratic convergence properties of Newton's method in
quasi-static non-linear analysis of solid structures it is crucial to obtain
accurate, algorithmically consistent tangent-stiffness matrices. A goal of the
study described in this paper was to establish the suitability of an
under-explored method for numerical computation of tangent-stiffness operators,
referred to as ``complex-step'', and compare the method with other techniques
for numerical derivative calculation: automatic differentiation, forward
finite-difference, and central finite-difference. The complex-step method was
implemented in a massively parallel computational peridynamics code for the
purpose of this comparison.  The methods were compared through invasive
profiling of the code for accuracy, speed, efficiency, and parallel
scalability. The research provides data that can serve as practical guide for
code developers and analysts faced with choosing which method best suits the
needs of their application code.  Additionally, motivated by the reproducible
research movement, all the of the code, examples, and workflow to regenerate
the data and figures in this paper are provided as open source.

\section{Re: W. Squire and G. Trapp (1998)}
\subsection{Abstract of prior work in field}

A method to approximate derivatives of real functions using complex variables
which avoids the subtractive cancellation errors inherent in the classical
derivative approximations is described. Numerical examples illustrating the
power of the approximation are presented.

\subsection{Aspects of CS investigated}

Establishing the accuracy of the method for computing first derivatives of
real functions through empirical tests, versus forward and central difference 
methods.

\subsection{Similarities with reviewed work}

The reviewed work uses the numerical differentiation method as shown in the
prior work. The accuracy of CS demonstrated in the prior work served as encouragement
for investigating CS further.

\subsection{Differences with reviewed work}

In the reviewed work the CS method is used to compute derivatives of a
non-linear vector function of vector variables wrt the components of the vector
variables, an extension of the method shown in the prior work.
Additionally, automatic differentiation is compared to CS. Beyond that, not
only is the accuracy of the method measured, but its impact as part of a Newton
Raphson solution scheme for a Peridynamics simulation run on a parallel
computer is investigated.

\subsection{Novelty of reviewed work}
The reviewed work examines applications of the CS method beyond taking derivatives alone,
whereas the prior work does not. 

\section{Re: A. Perez-Foguet, A. Rodriguez-Ferran, A. Huerta \emph{n.d. for local...} (2000)}
\subsection{Abstract of prior work in field}

In this paper, numerical differentiation is applied to integrate plastic
constitutive laws and to compute the corresponding consistent tangent
operators. The derivatives of the constitutive equations are approximated by
means of difference schemes. These derivatives are needed to achieve quadratic
convergence in the integration at Gauss-point level and in the solution of the
boundary value problem.  Numerical differentiation is shown to be a simple,
robust and competitive alternative to analytical derivatives. Quadratic
convergence is maintained, provided that adequate schemes and stepsizes are
chosen. This point is illustrated by means of some numerical examples.

\subsection{Aspects of CS investigated}

The prior work applies CS in computing derivatives needed for integrating
plastic stress, obtaining algorithmically consistent elastoplastic modulii, and
obtaining the Jacobian of the residual, for FEA simulations of two plastic
material models.

\subsection{Similarities with reviewed work}

First order CS is being used to compute the Jacobian of the residual in a solid mechanics.

\subsection{Differences with reviewed work}

The prior work demonstrates CS on a serial computer, and with a classical solid mechanics
formulation. Automatic differentiation is not investigated, and speed measurements are
not taken. Scalability studies with a variety of equivalent, progressively more refined meshes
are not done. Scalability as a function of distributed system size is not investigated.

\subsection{Novelty of reviewed work}

While the prior work does apply CS to computing Jacobians of the residual, and further applies CS
to computing elastoplastic modulii, the reviewed work is novel by contrast in offering discussion
of CS as applied to Peridynamics on a parallel computer, a larger range of problem parameters, 
CPU time data, and comparison to automatic differentiation as a competing method. 

\section{Re: A. Perez-Foguet, A. Rodriguez-Ferran, A. Huerta \emph{n.d. for nontrivial...}(2000)}
\subsection{Abstract of prior work in field}

In Reference [1] the authors have shown that numerical differentiation is a
competitive alternative to analytical derivatives for the computation of
consistent tangent matrices. Relatively simple models were treated in that
reference. The approach is extended here to a complex model: the MRS-Lade model
[2,3]. This plastic model has a cone-cap yield surface and exhibits strong
coupling between the flow vector and the hardening moduli. Because of this,
derivating these quantities with respect to stresses and internal variables
—the crucial step in obtaining consistent tangent matrices— is rather involved.
Numerical differentiation is used here to approximate these derivatives. The
approximated derivatives are then used 1) to compute consistent tangent
matrices (global problem) and 2) to integrate the constitutive equation at each
Gauss point (local problem) with the Newton-Raphson method. The choice of the
stepsize (i.e. the perturbation in the approximation schemes), based on the
concept of relative stepsize, poses no difficulties. In contrast to previous
approaches for the MRS-Lade model, quadratic convergence is achieved, for both
the local and the global problems. The computational efficiency (CPU time) and
robustness of the proposed approach is illustrated by means of several
numerical examples, where the major relevant topics are discussed in detail.

\subsection{Aspects of CS investigated}

The prior work applies CS in computing derivatives needed for integrating
plastic stress, obtaining algorithmically consistent elastoplastic modulii, and
obtaining the Jacobian of the residual, for FEA simulations of the MRS-Lade material
model.

\subsection{Similarities with reviewed work}

CS is used to compute the Jacobian of the residual. CPU runtime is measured.

\subsection{Differences with reviewed work}

The prior work demonstrates CS on a serial computer, and with a classical solid
mechanics formulation. Automatic differentiation is not investigated.
Scalability studies with a variety of equivalent, progressively more refined
meshes are not done. Scalability as a function of distributed system size is
not investigated.

\subsection{Novelty of reviewed work}

While the prior work does apply CS to computing Jacobians of the residual, and further applies CS
to computing elastoplastic modulii, the reviewed work is novel by contrast in offering discussion
of CS as applied to Peridynamics on a parallel computer, a larger range of problem parameters, 
and comparison to automatic differentiation as a competing method. 


\section{Re: J. Martins, P. Sturdza, J. Alonso (2003)}
\subsection{Abstract of prior work in field}

The complex-step derivative approximation and its application to numerical
algorithms are presented. Improvements to the basic method are suggested that
further increase its accuracy and robustness and unveil the connection to
algorithmic differentiation theory. A general procedure for the implementation
of the complex-step method is described in detail and a script is developed
that automates its implementation. Automatic implementations of the
complex-step method for Fortran and C/C++ are presented and compared to
existing algorithmic differentiation tools. The complex-step method is tested
in two large multidisciplinary solvers and the resulting sensitivities are
compared to results given by finite differences. The resulting sensitivities
are shown to be as accurate as the analyses. Accuracy, robustness, ease of
implementation and maintainability make these complex-step derivative
approximation tools very attractive options for sensitivity analysis.

\subsection{Aspects of CS investigated}

The prior work demonstrates that the CS method can be performed by using complex
datatypes for the variables of functions comprising an analysis code. The analysis code
can comprise a simulation or similar, and derivatives of response variables called
sensitivities can be computed. The accuracy of this approach is compared to automatic 
differentiation.

\subsection{Similarities with reviewed work}

CS is used to compute derivatives.

\subsection{Differences with reviewed work}

The prior work mentions employing CS within an iterative solver code within a sensitivity analysis, but
does not mention if CS is also used by the iterative solver itself. 

\subsection{Novelty of reviewed work}

The reviewed work is about applying CS to solving Peridynamics problems using Newton's method, and is therefor about an
application of the derivatives produced by CS. The prior work is about the application of CS itself to produce derivatives
which themselves are the object of interest. The reviewed work is novel by comparison since it examines aspects of CS
different to those examined in the prior work.

\section{Re: W. K. Anderson, J. C. Newman, D. l. Whitfield and E. J. Nielsen (2001)}
\subsection{Abstract of prior work in field}

The use of complex variables for determining sensitivity derivatives for
turbulent flows is examined. Although a step size parameter is required, the
numerical derivatives are not subject to subtractive cancellation errors and,
therefore, exhibit true second-order accuracy as the step size is reduced. As a
result, this technique guarantees two additional digits of accuracy each time
the step size is reduced one order of magnitude. This behavior is in contrast
to the use of finite differences, which suffer from inaccuracies due to
subtractive cancellation errors. In addition, the complex-variable procedure is
easily implemented into existing codes.

\subsection{Aspects of CS investigated}
Based on the accuracy of CS, the method is used to compute sensitivities in CFD
simulations. For the application of computing sensitivities, CS is compared to 
automatic differentiation.

\subsection{Similarities with reviewed work}
CS is used to compute derivatives and CS is compared to AD.

\subsection{Differences with reviewed work}
The prior work does not discus using the derivatives produced by CS to compute
the residual Jacobian, nor does it discuss the same for AD. The field is
fluid dynamics rather than solid mechanics, and the use of a prallel computer
is not discussed.

\subsection{Novelty of reviewed work}

The reviewed work offers discussion on the use of CS and AD to compute
residual Jcobians and the effect of doing so on the performance of Newton's
method in Peridynamics simulations run on a parallel computer, where the prior
work discusses the derivitives produced by CS and AD themselves as the objects 
of interest.

\section{A. H. Al-Mohy and N. J. Higham (2010)}
\subsection{Abstract of prior work in field}

We show that the Fréchet derivative of a matrix function f at A in the
direction E, where A and E are real matrices, can be approximated by Im f (A +
ihE)/ h for some suitably small h. This approximation, requiring a single
function evaluation at a complex argument, generalizes the complex step
approximation known in the scalar case. The approximation is proved to be of
second order in h for analytic functions f and also for the matrix sign
function. It is shown that it does not suffer the inherent cancellation that
limits the accuracy of finite difference approximations in floating point
arithmetic. However, cancellation does nevertheless vitiate the approximation
when the underlying method for evaluating f employs complex arithmetic.  The
ease of implementation of the approximation, and its superiority over finite
differences, make it attractive when specialized methods for evaluating the
Fréchet derivative are not available, and in particular for condition number
estimation when used in conjunction with a block 1-norm estimation algorithm.

\subsection{Aspect of CS discussed}
The prior work investigates the use of complex step in computing Frechet derivatives 
of matix to matrix functions.

\subsection{Similarities with reviewed work}
To compute the residual Jacobian in the peridynamics simulation dicussed in the reviewed
work is essentially computing a Frechet derivative of a higher dimension function of higher
dimension variables.

\subsection{Differences with reviewed work}
Where the application of the CS derivatives in the prior work is limited to a particular 
class of abstract functions to determine the method's suitability for use in condition number
estimation, the reviewed work applies CS to solving a mechanics problem.

\subsection{Novelty of reviewed work}
Because the reviewed work is not about just computing Frechet derivatives with CS but using
them in a numerical method to solve a set of equations describing a physical system, the reviewed
work is largely different from the prior work.

\section{W. Jin, B. H. Dennis, and B. P. Wang (2010)}
\subsection{Abstract of prior work in field}

The semi-analytical method (SAM) is a computationally efficient and easy to
implement approach often used for the sensitivity analysis of finite element
models.  However, it is known to exhibit serious inaccuracy for shape
sensitivity analysis for structures modeled by beam, frame, plate, or shell
elements. In the present paper, we use a semi-analytical approach based on
complex variables (SACVM) to compute the sensitivity of finite element mod- els
composed of beam and plate elements. The SACVM combines the complex variable
method (CVM) with the semi-analytical method (SAM) to obtain the response
sensitivity accurately and efficiently. The current approach maintains the
computational efficiency of the semi-analytical method but with higher
accuracy. In addition, the current approach is insensitive to the choice of
step size, a feature that simplifies its use in practical problems. The method
is applicable to any structural elements including beam, frame, plate, or shell
elements and only requires minor modifications to existing finite element
codes.

\subsection{Aspects of CS discussed}
The use of CS to compute sensitivities is disucssed in the contect of a solid mechanics
FEA analysis.

\subsection{Similarities with reviewed work}
CS is used to compute derivitives.

\subsection{Differences with reviewed work}
In the prior work the derivitives produced by CS themselves are the object of interest, where
the reviewed work uses CS in a numerical method for solving a non-linear system.

\subsection{Novelty of reviewed work}
The prior work and the reviewed work discuss different aspects of the CS method. The
reviewed work discusses using the CS in a numerical method to solve a non-linear system,
where the prior work discusses the derivatves produced by CS as an object of interest.

\section{A. Voorhees, H. Millwater, and R. Bagley (2011)}
\subsection{Abstract of prior work in field}

Shape sensitivity analysis of finite element models is useful for structural
optimization and design modifications. Complex variable methods for shape
sensitivity analysis have some potential advantages over other methods. In
particular, for first order sensitivities using the complex Taylor series
expansion method (CTSE), the implementation is straightforward, only requiring
a perturbation of the finite element mesh along the imaginary axis. That is,
the real valued coordinates of the mesh are unaltered and no other
modifications to the software are required. Fourier differentiation (FD)
provides higher order sensitivities by conducting an FFT analysis of multiple
complex variable analyses around a sampling radius in the complex plane.
Implementation of complex variable sensitivity methods requires complex
variable finite element software such that complex nodal coordinates can be
used to implement a perturbation in the shape of interest in the complex
domain. All resulting finite element outputs such as displacements, strains and
stresses become complex and accurate derivatives of all finite element outputs
with respect to the shape parameter of interest are available. The
methodologies are demonstrated using two-dimensional finite element models of
linear elasticity problems with known analytical solutions. It is found that
the error in the sensitivities is primarily defined by the error in the finite
element solution not the error in the sensitivity method. Hence, more accurate
sensitivities can be obtained through mesh refinement.

\subsection{Aspect of CS discussed}
The prior work dicusses the background of the CS method and examines it as a subset
of Fourier Differentiation. CS and Fourier Differentiation are then applied to 
computing sensitivities within a solid mechanics FEA.

\subsection{Similarities with reviewed work}
CS is used to compute derivatives.

\subsection{Differences with reviewed work}
In the prior work, CS is discussed and related to a broader class of numerical methods. CS is then used
to compute sensitivities. In the reviewed work, complex step is used to compute residual Jacobians
which are used in Newton's method for solving non-linear systems.

\subsection{Novelty of reviewed work}
The reviewed work and prior work discuss different aspects of CS in that they examine different
applications of the method.

\section{K.-L. Lai and J. L. Crassidis (2008)}
\subsection{Abstract of prior work in field}

A general framework for the first and second complex-step derivative
approximation to compute numerical derivatives is presented.  For first
derivatives the complex-step approach does not suffer roundoff errors as in
standard numerical finite-difference approaches.  Therefore, since an
arbitrarily small step size can be chosen, the complex-step approach can
achieve near analytical accuracy.  However, for second derivatives straight
implementation of the complex-step approach does suffer from roundoff errors.
Therefore, an arbitrarily small step size cannot be chosen. In this paper the
standard complex-step approach is expanded by using general complex-step sizes
to provide a wider range of accuracy for both the first- and second-derivative
approximations. Even higher accuracy formulations are obtained by repetitively
applying Richardson extrapolations. The new extensions can allow the use of one
step size to provide optimal accuracy for both derivative approximations.

\subsection{Aspects of CS discussed}
First and second order CS methods for computing derivatives. Choosing optimal
step size.

\subsection{Similarities with reviewed work}
Derivatives are computed with CS.

\subsection{Differences with reviewed work}
The prior work discusses computing first and second order accurate derivatives with CS and choosing optimal step size.
The reviewed work discussing using the derviatives computed with the CS method in a numerical method for solving 
a non-linear system.

\subsection{Novelty of reviewed work}
The reviewed work is offers discusion of the CS method as used in a numerical method for solving a non-linear system, where 
the prior work dicusses the derivatives produced by CS themselves as an object of interest.

\section{R. Abreu, D. Stich, and J. Morales (2013)}
\subsection{Abstract of prior work in field}

We generalize the well known Complex Step Method for computing derivatives by
introducing a complex step in a strict sense. Exploring different combinations
of terms, we derive 52 approximations for computing the first order derivatives
and 43 for the second order derivatives. For an appropriate combination of
terms and appropriate choice of the step size in the real and imaginary
directions, fourth order accuracy can be achieved in a very simple and
efficient scheme on a compact stencil. New different ways of computing second
order derivatives in one single step are shown. Many of the first order
derivative approximations avoid the problem of subtractive cancellation
inherent to the classic finite difference approximations for real valued steps,
and the superior accuracy and stability of the generalized complex step
approximations are demonstrated for an analytic test function.

\subsection{Aspects of CS discussed}
Computing derivatives with CS is discussed. Combinations of real and imaginary step components
are studied to determine CS methods with extended accuracy order.

\subsection{Similarities with reviewed work}
Derivatives are computed with CS.

\subsection{Differences with reviewed work}
The prior work discusses using the CS method to compute derivatives and attempts to extend the
methods for that application. The reviewed work us the derivatives computed with CS in a numerical
method for solving a non-linear system.

\subsection{Novelty of reviewed work}
The prior work discusses the derviatives computed by the CS method as an object
of interest, while the reviewed work discusses using the derivatives computed
by the CS method in a numerical method for solving a non-linear system.
Because the the reviewed work and the prior work discuss different aspects of
CS and different applications, the reviewed work is novel.

\section{J. N. Lyness and C. B. Moler (1967)}
\subsection{Abstract of prior work in field}

Many algebraic computer languages now include facili- ties for the evaluation
of functions of a complex variable. Such facilities can be effectively used for
numerical differentiation. The method we de- scribe is appropriate for
computing the derivatives f(n) (x) of any analytic function which can be
evaluated at points in the complex plane near x.  Such a function might be a
complicated rational combination of elementary functions of x which is
difficult to differentiate analytically.  In this paper we derive several
formulas, any of which is suitable for evaluating the nth derivative of a
complex analytic function at a point in terms of function evaluations at
neighborimg points. For simplicity we assume throughout this paper that the
derivatives are to be evaluated at the origin. In this Introduction we state
one formula; in succeeding sections we prove this and related formulas, discuss
their degree and the error in their application, and give an example.

\subsection{Aspect of CS discussed}
The CS method is identified in an exact form.

\subsection{Similarities with reviewed work}
Derivatives are computed with the CS method.

\subsection{Differences with reviewed work}
The prior work develops the CS method, while the reviewed work
is about an application of the CS method.

\subsection{Novelty of reviewed work}
Because the reviewed work discusses an application the material presented in the
prior work not discussed in the prior work, it introduces novel material relative to the
prior work.

\section{J. N. Lyness (1968)}
\subsection{Abstract of prior work in field}

In a previous paper (Lyness and Moler [1]), several closely related formulas of
use for obtaining a derivative of an analytic function numerically are derived.
Each of these formulas consists of a convergent series, each term being a sum
of function evaluations in the complex plane.  In this paper we introduce a
simple generalization of the previous methods; we investigate the "truncation
error" associated with truncating the infinite series.  Finally we recommend a
particular differentiation rule, not given in the previous paper.

\subsection{Aspects of CS discussed}
The series representation derived CS method is shown in its truncated form.

\subsection{Similarities with reviewed work}
Derivatives are computed with the CS method.

\subsection{Differences with reviewed work}
The prior work develops the CS method, while the reviewed work
is about an application of the CS method.

\subsection{Novelty of reviewed work}
Because the reviewed work discusses an application the material presented in the
prior work not discussed in the prior work, it introduces novel material relative to the
prior work.

\end{document}
