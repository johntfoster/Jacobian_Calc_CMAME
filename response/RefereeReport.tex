Page 1

Computer Methods in Applied Mechanics and Engineering (CMAME) \\Referee
Repor t \\

Title: ``A comparison of different methods for calculating
tangent-stiffness matrices in a massively \\parallel computational
peridynamics code'' \\

Authors: M. D. Brothers, J. T. Foster, and H. R. Millwater \\

Manuscript number: CMAME-D-13-01004 \\

A. Summary \\

This manuscript presents comparisons between different numerical
differentiation methods \\for the computation of entries of
tangent-stiffness matrices, used in Newton's method for \\quasi-static
nonlinear analysis of solid structures. The methods include: automatic
differen- \\tiation (AD), forward finite difference (FD), centered finite
difference (CD) and complex-step \\(CS). The authors perform the
comparisons within the computational peridynamics code \\Peridigm, where
the authors implemented the CS method. Numerical results include com-
\\parisons of computational times between all the methods, as well as
comparisons of accu- \\racy between FD, CD, and CS, relative to AD. The
results show that AD is the fastest method \\in serial and FD in
parallel, whereas CS is the slowest method. However, the accuracy of
\\CS is several orders of magnitude higher than the finite difference
methods (FD and CD). \\Consequently, although AD seems to be the best
method in terms of combined speed and \\accuracy, the results suggest
that CS may represent a viable alternative for cases where AD
\\implementations are not practical. \\B. Evaluation \\

This manuscript focuses on the CS method. This method and its comparison
to other finite- \\difference methods have been previously presented in
the literature. For example, the CS \\method has been described in
{[}1{]}, with a comparison between FD and CS. Applications of \\the CS
method to compute tangent matrices appear in {[}2, 3{]}, with
comparisons between \\FD, CD, and CS. In addition, the CS method has
been used for sensitivity analysis in {[}5, 6{]}, \\and in {[}4{]}, a
detailed description of the CS method and comparisons to FD, CD, and AD
are \\presented. Fur ther uses of the complex-step method appear in
{[}7, 8, 9{]}, and extensions of \\the CS method are discussed in {[}10,
11{]}. \\However, this manuscript presents some aspects of the CS method
which seem novel. \\These are: \\

1. comparisons between FD, CD, CS, and AD in terms of computational
time, \\2. implementation of the CS method in parallel, \\3.
implementation of the CS method within Peridigm. \\

In par ticular, the implementation in Peridigm is of strong interest to
researchers working with \\the peridynamic theory of solid mechanics.
\\I believe this manuscript is of value to both the computational
mechanics community in gen- \\eral, and the peridynamic community in par
ticular. \\I suggest it to be considered for pub- \\lication in the
Computer Methods in Applied Mechanics and Engineering journal, after the
\\issues described below have been properly addressed. \\

1 \\

Page 2

C. Information to the authors \\

I. Larger issues \\

1. Some par ts of the manuscript suggest that the authors introduced a
new method, \\i.e., the complex-step (CS) method. For example, in the
introduction, they say \\``A goal of this study was to develop and
evaluate a new, accurate, and practical \\method \ldots{}'' Another
example appears in page 3, where the authors say ``The aim \\of this
paper is not only to introduce the new complex-step method in the
context \\of evaluating tangent-stiffness matrices ..''. However, the CS
method has already \\appeared in many publications since 1998 {[}1{]}.
Please revise the way you refer to \\the CS method along the manuscript.
\\2. The work presented focuses on a parallel computational peridynamic
code, and \\uses peridynamic models for mechanics. However, no detailed
explanation of the \\equations involved is presented. Although the
tangent-stiffness calculations are \\not specialized to Peridigm, the
authors should include a shor t section introducing \\the peridynamic
theory and the specific constitutive model used, i.e., the elastic
\\peridynamic solid. You may add this to the already included very brief
description \\of peridynamics in page 9 (lines 15-26). This will also
help to explain the concept \\of the peridynamic horizon in page 10.
\\3. The manuscript seems to be written in a language more familiar to
computer scien- \\tists. I would encourage the authors to revise cer
tain par ts of the manuscript to be \\presented in a language more
familiar to the computational mechanics community. \\4. In Section
1.1.1, please include a more comprehensive literature review for the CS
\\method, including fur ther references (some of them are listed below).
\\In par ticu- \\lar, specify what aspects of the CS method have been
investigated in each refer- \\ence. Then, you need to clearly mention
your contribution within the CS method, \\in contrast to the other
references (some comments in this regard appear in the \\manuscript in
page 8). \\5. The comparisons of the different algorithms are only
performed with respect to the \\construction of tangent-stiffness
matrices. In par ticular, the authors always use the \\solution of AD at
each time step to compute the entries of the tangent-stiffness
\\matrices, even for the FD, CD, and CS methods. However, the authors
need to \\demonstrate the performance of the CS method, in comparison to
the other ones, \\with respect to the solution of real, ideally
non-linear problems. In other words, how \\the solutions compare in
terms of accuracy and speed between the methods when \\solving given
problems? Please include two examples of interest. The authors \\suggest
that two examples involving non-linear systems are already included in
the \\paper repository. \\The authors say in the Conclusion: ``While the
results showed that CS produced \\more accurate tangent-stiffness
matrices than CD and FD under the parameters \\of the tests, it was not
determined whether or not this is a clear advantage of CS \\over FD in
terms of accuracy of final predicted displacements and speed of conver-
\\gence.'' This needs to be studied and results in this regard added to
this manuscript. \\II. Smaller issues \\

1. Pg. 2: In line 14, explain what does the term ``algorithmically
consistent'' mean. \\2. Pg. 2: In line 36, what do you mean by
``discrimination''? \\3. Pg. 2: In line 37, explain what do you mean by
``in-situ instrumentation''. \\4. Pg. 3: In lines 19-29, specify the
sections where each par t of the manuscript is \\included. \\

2 \\

Page 3

5. Pg. 3: You may consider including the exact web address of the paper
repository. \\6. Pg. 3: Please include the forward and centered finite
difference expressions for \\easier comparison with (2); include the
truncation error as well. You do not have to \\derive these expressions
though. \\7. Pg. 4: In Eq. (1), `` ∂ f 2 \\∂x2 '' should be `` ∂ 2 f
\\∂x2 ''; similarly, for the third derivative. \\8. Pg. 4: You may
consider writing Eq. (2) with the truncation error, i.e., \\

+ O(h2 ). \\

(2) \\

Im(f (x + ih)) \\h \\

∂ f (x) \\= \\∂x \\In this case, one replaces ``≈'' by ``=''. \\9. Pg.
6: In lines 9-11, please add references to the ``Poisson problem'' and
to the \\``mathematical optimization''. \\10. Pg. 7: Should Eqs. (4),
(5), and (6) use ``≈'' instead of ``=''? \\11. Pg. 7: In line 28, note
that (cid:126)u is not the current configuration, but the displacement
\\field. Please revise the sentence. \\12. Pg. 7: Please provide an
expression for F int \\in the model used. \\i \\13. Pg. 9: In line 26,
please add the reference {[}12{]} below to ``coupling to molecular
\\simulations {[}29{]}.'' \\14. Pg. 10: Please add an illustration of
the block undergoing tension in Section 2.2, \\for clarity. Also,
clarify the displacement boundary conditions; are they applied in a
\\volumetric layer? \\15. Pg. 10: In line 40, you mentioned a bandwidth
of 7. This seems to be clear in 1D. \\Does this hold in higher
dimensions? Please clarify. \\16. Pg. 10: In lines 47-49, you say ``The
specific parameters \ldots{} can be found \ldots{} in the \\paper
repository.'' It seems a good idea to have the paper repository with
files that a \\user can eventually use. However, the content of the paper
should be independent \\of the repository. Please include a table with
the parameters used. Similarly, in \\page 11 (lines 11-13). \\17. Pg.
14: In line 55, is there any conclusion you can draw from the power law
index \\1.0 × 10−6? \\18. Pgs. 14-: Please clarify the value of the
meshsize used on each of the experiments. \\19. Pg. 16: Would it be
possible to rescale Fig. 2, so that the dependence of the time \\on the
number of cores will look indeed linear? \\20. Pg. 16: In lines 54-56,
you say ``It should be pointed out that compared to the (cid:96)2 -
\\norms of any of the TS matrices by themselves, the magnitude of the
accuracy \\metric \ldots{} is relatively small.'' Please include in the
manuscript the explicit informa- \\tion on the relative error. \\It
seems that you already have such information in the \\paper repository.
\\21. Pg. 17: Include the values of the probe distances h used for each
method. \\22. Pg. 17: In lines 53-54, you say ``Accuracy trends matched
those for the single core \\tests.'' Indeed, comparing Figs. 3 and 4,
the results for FD seem to be quite the \\same in serial and in
parallel. For CS the accuracy seems higher in serial. For CD, \\the
accuracy is also higher in serial and, in serial, it has a fluctuating
behavior. Can \\you comment on these observations? \\23. Pgs. 17-18:
What are the units of K ? If F int in Eq. (4) etc. is internal force,
then K \\should have units of force/length. However, in Figs. 3 and 4
you use the unit of \\MPa, which is force/length2 . Please clarify.
\\24. Pg. 19: Are the units of the ver tical axis in Fig. 5 ``seconds'',
as in Figs. 1 and 2? \\Please indicate that. Also, is the label of the
ver tical axis in Fig. 6 correct? \\

3 \\

Page 4

25. Pg. 20: In lines 32-35, clarify what ``byte-copyable'' mean and why
is that impor tant. \\26. Pg. 21: Please include a section title for the
references. \\27. Pg. 24: In line 23, you may add the truncation error
O(h2 ) to the expression. Oth- \\erwise, you may need to replace the
``='' sign by the ``≈'' sign. However, adding \\the error term would
make more clear the limiting expression when h → 0. Also, \\replace the
comma by a period. \\28. Pg. 24: Why do we need S and X in Appendix B?
\\29. Pgs. 25-26: Are the functions in the second column of the table in
2(c) (page 25) \\approximations? Why do you label the column
``Approximates Function''? Simi- \\larly, in line 16 (page 26) you say
``approximated by the par tial derivative''; is it an \\approximation?
\\30. Pg. 26: Please clarify whether storage is needed in automatic
differentiation. For \\example, do we need to store w, v , and u? \\III.
Typos/editing suggestions \\

is \\

1. Pg. 1: In line 43, remove ``the'' in ``all the of the code''. \\2.
Pg. 1: In line 46, replace ``numeric differentiation'' by ``numerical
differentiation''. \\3. Pg. 2: In line 25, the sentence ``The
distinction of ``accurate'' is defined by \ldots{}'' \\awkward. Please
revise it. \\4. Pg. 2: In lines 32 and 48, ``finite-difference'' should
be ``forward-difference''. \\5. Pg. 2: In lines 33, I believe ``aide''
should be ``aid''. \\6. Pg. 3: In line 12, I would replace ``such as
required'' by ``such as those required''. \\7. Pg. 3: In line 22,
replace ``with each the methods'' by ``with each of the methods''. \\8.
Pg. 3: In line 25, replace ``implementing complex-step'' by
``implementing the complex- \\step method''. \\9. Pg. 3: I would replace
the title of Section 1.1, ``Differentiation Techniques'' by ``Dif-
\\ferentiation techniques'' for consistency, i.e., only capitalize the
first letter of the first \\word. Similarly, for the rest of the section
titles. \\10. Pg. 4: In line 10, ``be be made complex'' should be ``be
made complex''. \\11. Pg. 4: In Eq. (1), replace comma by period. \\12.
Pg. 4: In line 22, you have an extra white space in ``O (h2 )''. \\13.
Pg. 4: I would check the proper use of hyphens along the manuscript. For
example, \\in the title of Section 1.1.2, I would replace
``Automatic-differentiation'' by ``Automatic \\differentiation'' (i.e.,
no hyphen). \\14. Pg. 4: In line 42, replace ``based the chain-rule'' by
``based on the chain rule''. \\15. Pg. 4: In lines 45-46, ``(add,
multiply, power, transcendental and the like)'' should \\probably be
``(addition, multiplication, exponentiation, and the like)'' or similar.
\\16. Pg. 4: In line 47, I would replace ``used in the study'' by ``used
in this study''. Similarly, \\in other instances of the manuscript.
\\17. Pg. 5: In line 39, you may replace ``the par tials'' by ``the par
tial derivatives''. Similarly \\in page 26. \\18. Pg. 5: I would replace
the title of Section 1.2 ``Tangent-stiffness'' by ``Tangent-stiffness
\\matrix''. Similarly, in line 50. \\19. Pg. 6: In line 14,
``quasi-Newton's method's'' should probably be ``quasi-Newton
\\methods''. \\20. Pg. 6: In line 18, ``about about a par ticular''
should be ``about a par ticular''. \\21. Pg. 6: In line 51, you may
replace ``argument to F '' by ``argument of F ''. \\22. Pg. 7: In line
11, you may replace ``single independent vector component of the
\\function'' by ``single independent vector component of the argument of
the function''. \\

4 \\

Page 5

23. Pg. 7: In line 24, replace ``Where Kij is'' by ``where Kij is''.
\\24. Pg. 7: In line 40, replace ``in literature {[}19{]}'' by ``in the
literature {[}19{]}''. \\25. Pg. 8: Remove the period at the end of the
titles of Sections 2 and 2.1, for consis- \\tency. Similarly, for the
rest of the manuscript. Also, use italic font for ``Peridigm'' in \\the
title of §2.1. \\26. Pg. 9: In line 29, replace ``Perdigm'' by
``Peridigm''. Similarly in page 13 (line 44). \\27. Pg. 9: In line 37,
replace ``be simply be declared'' by ``be simply declared''. \\28. Pg.
9: In line 55, I would replace ``computational scientists'' by
``computer scientists''. \\29. Pg. 10: In line 30, I would replace ``The
purpose of applying the load'' by ``Applying \\the load''. \\30. Pg. 13:
I would add a comma at the end of Eq. (8), and replace ``Where D is
\\distance'' by ``where D is distance'' in line 14. \\31. Pg. 13: In
line 15, replace ``based method, M is'' by ``based method, and M is''.
\\32. Pg. 13: In line 31, replace ``correspond to'' by ``corresponds
to''. \\33. Pg. 14: In lines 19-21, you have twice the word ``nonzero''
in ``number of nonzero \\tangent-stiffness (TS) non-zero matrix
elements''. \\34. Pg. 14: In lines 22, I would replace ``The units for
accuracy are derived equation (8) \\and the units used to define bulk
material proper ties, mentioned in Section 2.2.'' by \\``The accuracy
measure is given by equation (8) and the choice of elastic moduli is
\\mentioned in Section 2.2.'' \\35. Pg. 15: In line 54, it seems that
``yet still fall'' should be ``yet still falls''. \\36. Pg. 17: In line
45 you say ``h, cannot be too small,''. \\I understand that you mean
\\that you can take h as small as possible. However, the way written may
have the \\opposite connotation. I would revise it for clarity. \\37.
Pg. 18: In lines 35-36, replace ``the number nonzero'' by ``the number
of nonzero''. \\38. Pg. 18: In line 39, replace ``the the number of
nonzero'' by ``the number of nonzero''. \\39. Pg. 18: In line 50, I
would replace ``experimental results,'' by ``computational experi-
\\ments,''. Otherwise, it may give the impression of physical
experiments. \\40. Pg. 20: In line 13, is the use of the word ``order''
in ``number of iterations taken order \\to measure'' correct? \\41. Pg.
20: In lines 26-27, replace ``complex-step all of the methods'' by
``complex-step \\in all of the methods''. \\42. Pg. 20: In line 50, you
may either replace ``to implement; It only'' by ``to implement; \\it
only''; or replace the semicolon by a period. \\43. Pgs. 21-23: You may
capitalize the first letters on the journal names in your refer- \\ences
{[}9, 20{]} \\44. Pg. 22: In your reference {[}10{]}, you may replace
``fr ´echet'' by ``Fr ´echet''. \\45. Pg. 23: Update your reference
{[}21{]} (see the reference {[}3{]} in this repor t). Also, \\update
your reference {[}27{]}. \\46. Pg. 24: In line 18, replace the comma by
a period. \\47. Pg. 24: I would replace the comma by a period in line
28, and replace ``using \\L'H ˆopital's rule'' by ``Using l'H ˆopital's
rule'' in line 31. \\48. Pg. 24: I would replace the comma by a period
in line 34, and replace ``of course, \\as h → 0'' by ``As h → 0'' in
line 37. \\49. Pg. 24: I would replace the title of Appendix B ``AD
Example'' by ``Automatic differ- \\entiation example''. \\50. Pg. 24: In
line 45, you may replace ``sin(cos(x))'' by ``sin(cos(x))''. For that,
use \textbackslash{}sin \\instead of sin, etc. (similarly in the bottom
of the page and in page 25). Also, add \\a period at the end of the
expression. \\

5 \\

Page 6

51. Pg. 25: In line 55, you may remove the ``·'' symbol in `` d \\dx ·
g(h(k(x)))''. \\52. Pgs. 25 and 26: I would number the table in 2(c)
(page 25). Let's assume the \\number is ``2''. Then, in line 13 (page
26), I would replace ``chosen for each function \\and held as constant''
by ``chosen for each function as in Table 2 and held constant''. \\53.
Pg. 26: In line 24, add a period at the end of the expression. \\

References \\

{[}1{]} W. Squire and G. Trapp, Using complex variables to estimate
derivatives of real functions, \\SIAM Rev. 40 (1998), pp. 110-112. \\

{[}2{]} A. P ´erez-Foguet, A. Rodr´ıguez-Ferran, A. Huer ta, Numerical
differentiation for local and \\global tangent operators in
computational plasticity, Comput. Methods Appl. Mech. Engrg. \\189
(2000), pp. 277-296. \\

{[}3{]} A. P ´erez-Foguet, A. Rodr´ıguez-Ferran, A. Huer ta, Numerical
differentiation for non-trivial con- \\sistent tangent matrices: an
application to the MRS-Lade model, Int. J. Numer. Meth. Engng \\48
(2000), pp. 159-184. \\

{[}4{]} J. Mar tins, P. Sturdza, J. Alonso, The complex-step derivative
approximation, ACM Transac- \\tions on Mathematical Software (TOMS),
ISSN 0098-3500, 09/2003, Volume 29, Issue 3, pp. \\245-262. \\

{[}5{]} W. K. Anderson, J. C. Newman, D. L. Whitfield, and E. J. Nielsen,
Sensitivity analysis for \\Navier--Stokes equations on unstructured
meshes using complex variables, AIAA Journal, \\ISSN 0001-1452, 01/2001,
Volume 39, Issue 1, pp. 56-63. \\

{[}6{]} J. C. Newman, W. K. Anderson, and L. D. Whitfield,
Multidisciplinary sensitivity derivatives \\using complex variables,
Tech. Rep. MSSU-COE-ERC-98-08 (July), Computational Fluid Dy- \\namics
Laboratory. \\

{[}7{]} A. H. Al-Mohy and N. J. Higham, The complex step approximation
to the Fr ´echet derivative \\of a matrix function, Numer. Algor. 53
(2010), pp.133-148. \\

{[}8{]} W. Jin, B. H. Dennis, and B. P. Wang, Improved sensitivity
analysis using a complex variable \\semi-analytical method, Struct.
Multidisc. Optim. 41 (2010), pp. 433-439. \\

{[}9{]} A. Voorhees, H. Millwater, and R. Bagley, Finite Elements in
Analysis and Design 47 (2011), \\pp. 1146-1156. \\

{[}10{]} K.-L. Lai and J. L. Crassidis, Extensions of the first and
second complex-step derivative ap- \\proximations, Journal of
Computational and Applied Mathematics 219 (2008), pp. 276-293. \\

{[}11{]} R. Abreu, D. Stich, and J. Morales, On the generalization of
the complex step method, Journal \\of Computational and Applied
Mathematics 241 (2013), pp. 84-102. \\

{[}12{]} P. Seleson, M. L. Parks, and M. Gunzburger, Peridynamic
state-based models and the \\embedded-atom model. Communications in
Computational Physics, 15(1) (2014), pp. 179- \\205. \\

6 \\

Page: \hyperref[1]{1}, \hyperref[2]{2}, \hyperref[3]{3},
\hyperref[4]{4}, \hyperref[5]{5}, \hyperref[6]{6}
